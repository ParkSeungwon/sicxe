\documentclass[12pt,a4paper]{article}
\synctex=1
\usepackage[utf8]{inputenc}
\usepackage[margin=1cm,bottom=2cm]{geometry}
\usepackage{graphicx}
%\usepackage{verbatim}
\usepackage{listings}
\usepackage{multicol}
\usepackage{libertine}
\usepackage{pgfornament}
\usepackage{eso-pic}
\usepackage{textcomp}
\usepackage{courier}
\usepackage[hangul]{kotex}
\linespread{1.3}

\title{
	\centering
	\pgfornament[width=12cm,color=teal]{84}\\
	\vspace{1cm}
	\fontsize{50}{50} \selectfont {시스템 S/W 실습9}\\
	\pgfornament[width=12cm,color=teal]{88}\\
	\vfill}
\author{
	\LARGE
	\begin{tabular}{rl}
		\hline
		학번 : & 2016110056\\ 
		학과 : & 불교학부 \\
		이름 : & 박승원\\
		날짜 : & \today\\
		\hline
	\end{tabular}\vspace{2cm}
	\\
	\includegraphics[width=0.5\textwidth]{/home/zezeon/Dropbox/Photos/logo.jpg}
}
\date{}


\begin{document}
\maketitle
\newpage
\noindent
\lstset{columns=flexible, tabsize=4, frame=single, showstringspaces=false, breaklines=true, upquote=true}

\pagenumbering{gobble}
\lstset{language=C++}
%\begin{multicols}{2}
1. 다음에 주어진 리눅스 명령을 입력하고, 실습하시오.
1)환경변수: 문자열변수 : \$HOME, \$PATH, \$MAIL, \$USER, \$TERM, \$SHELL\\
\$echo HOME=\$HOME, PATH=\$PATH

2)내장변수: \$\$(셸의 프로세스 id), \$0(셀 스크립트 이름), \$1...\$9(명령어 줄 인수 참조), \$*(모든 명령어 줄 인수의 목록), \$@(\$*의 변형)


\lstinputlisting[caption=sum.c]{sum.c}
\lstinputlisting[caption=sum.s]{sum.s}

\lstinputlisting[caption={objdump -d sum.o}]{sum.dump}
\vspace{1cm}

2. 다음에 주어진 main.c를 컴파일하여 main.o를 얻고, main.o와 sum.o를 link하여 실행파일 prog를 얻고, prog로서 Linux debugger program인 'gdb'를 사용해서 1명령씩 실행하면서 필요한 register들의 값을 확인하시오.

\lstinputlisting[caption=main.c]{main.c}
\lstinputlisting[caption={gdb 실행}]{log}

{\Huge소감}
\indent
gdb나 컴파일러가 아직도 모르는 옵션들이 매우 많다는 것을 느꼈다.  

\end{document}
