\documentclass[11pt,a4paper,twocolumn,landscape]{article}
\synctex=1
\usepackage[utf8]{inputenc}
\usepackage[margin=1cm,left=0.5cm]{geometry}
\usepackage{graphicx}
%\usepackage{verbatim}
\usepackage{listings}
\usepackage{textcomp}
\usepackage{courier}
\usepackage[hangul]{kotex}
\linespread{1}

\begin{document}
\pagenumbering{gobble}
\begin{center}
	\Huge시스템 소프트웨어와 실습  과제\\
	\vspace{2cm}
\hfill\includegraphics[height=30pt]{logo.jpg}

\hfill\Large 2016110056 불교학부 박승원

\hfill\today
\end{center}

\noindent
\lstset{columns=flexible, tabsize=4, frame=single, showstringspaces=false, breaklines=true, upquote=true, basicstyle=\ttfamily\scriptsize}
\begin{enumerate}
\lstset{language=[x86masm]Assembler}
\lstinputlisting{1.s}
\lstinputlisting{1.o}
	\lstinputlisting{2.s}
		\lstinputlisting{2.o}
		\lstinputlisting{1.txt}
		\lstset{language=C}
직접 수업시간에 들은 내용을 토대로 컴파일러와 실행인터프리터를 구현해 보았다.
SIC시뮬레이터와 동일한 결과를 얻었다.
그 소스 코드를 첨부한다.
\lstinputlisting{sic.h}	
\lstinputlisting{sic.cc}
\lstinputlisting{compiler.h}
\lstinputlisting{compiler.cc}
\lstinputlisting{interpreter.h}
\lstinputlisting{interpreter.cc}
\lstinputlisting{compile.cpp}
\lstinputlisting{run.cpp}


\end{enumerate}

{\Huge소감}
\indent
SIC 에뮬레이터의 연습을 통해 컴퓨터의 작동원리를 파악할 수 있었다.
에뮬레이터는 굉장히 어려운 프로그래밍으로 생각했고, 하드웨어적인 지식이 많아야 
하는 것으로 알고 있었는데, 
직접 컴파일러를 작성해 보니 에뮬레이터가 생각보다 쉽게 구현이 되는 것에 놀랐다.
하드웨어가 단순한 원리에 의해서 움직인다는 교수님의
말씀을 이해할 수 있었다. 


수업을 통해 정확한 지식을 얻고 작성해 보니, 어렴풋이 
알고 있던 것들이 명확해졌고, 매우 성취감이 컸다. 
지금은 아직 조건분기 등의 구현과 문자 데이터의 처리등은 만들지 않았지만,
수업의 진도 대로 차근차근 만들어 보도록 해야겠다.
클래스를 만들며 수업을 들으니 SIC의 작동원리를 구체적으로 이해할 수 있게 되는 것 같다.
\end{document}
