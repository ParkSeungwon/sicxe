\documentclass[12pt,a4paper]{article}
\synctex=1
\usepackage[utf8]{inputenc}
\usepackage[margin=1cm,bottom=2cm]{geometry}
\usepackage{graphicx}
%\usepackage{verbatim}
\usepackage{listings}
\usepackage{multicol}
\usepackage{libertine}
\usepackage{pgfornament}
\usepackage{eso-pic}
\usepackage{textcomp}
\usepackage{courier}
\usepackage[hangul]{kotex}
\linespread{1.3}

\title{
	\centering
	\pgfornament[width=12cm,color=teal]{84}\\
	\vspace{1cm}
	\fontsize{50}{50} \selectfont {시스템 S/W 실습6\\선택 과제}\\
	\pgfornament[width=12cm,color=teal]{88}\\
	\vfill}
\author{
	\LARGE
	\begin{tabular}{rl}
		\hline
		학번 : & 2016110056\\ 
		학과 : & 불교학부 \\
		이름 : & 박승원\\
		날짜 : & \today\\
		\hline
	\end{tabular}\vspace{2cm}
	\\
	\includegraphics[width=0.5\textwidth]{/home/zezeon/Dropbox/Photos/logo.jpg}
}
\date{}


\begin{document}
\maketitle
\newpage
\noindent
\lstset{columns=flexible, tabsize=4, frame=single, showstringspaces=false, breaklines=true, upquote=true}

\pagenumbering{gobble}
\lstset{language=C++}
%\begin{multicols}{2}
2. SIC srcfile을 읽어서 각 줄을 LABEL, OPCODE, OPERAND 로 분리하여 Intfile에 출력하면서 SYMTAB[]을 생성한다.(실습 5의 PASS1 내용), 
그리고 Intfile을 입력하고, SYMTAB[]의 내용을 사용하여 OPCODE가 기계명령이면 {LOCCTR, CODE, ADDRESS}를 ObjTmpfile에 출력하는 어셈블러 PASS2에 해당하는 프로그램을 구현하고 실습하시오.
LOCCTR은 Location, Counter이고, CODE는 OPTAB[]에서 읽은 기계코드이고, ADDRESS는 OPERAND로서 SYMTAB[]에서 읽은 값이다.

\vspace{1cm}
단, C program compile 명령은 다음과 같고, 입력파일 srcfile은 각자 준비한다.

\$g++ -o pass2 pass2.c\\

\$./pass2 srcfile Intfile ObjTmpFile

\lstinputlisting[caption={실행 파일의 소스, pass2.cpp}]{pass2.cpp}
\lstinputlisting[caption={임의의 srcfile}]{1.s}
\lstinputlisting[caption={소오스와 거의 차이가 없는 Intfile}]{int}
실제로는 vector에 분리하여 넣었지만, 파일 상으로는 소오스와 거의 차이가 없다.

\lstinputlisting[caption={결과로 나온 ObjTmpFile}]{obj}
결과가 주소와 opcode, operand의 주소를 제대로 찾고 있음을 알 수 있다.

{\Huge소감}
\indent
기존에 냈던 소스들을 약간 수정하여 합쳤습니다. 

\end{document}
